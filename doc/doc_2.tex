
%----------------------------------------------------------------------------------------
%	PACKAGES AND OTHER DOCUMENT CONFIGURATIONS
%----------------------------------------------------------------------------------------

\documentclass[paper=a4, fontsize=11pt]{scrartcl} % A4 paper and 11pt font size

\usepackage[utf8]{inputenc}
\usepackage[T1]{fontenc} % Use 8-bit encoding that has 256 glyphs
\usepackage[polish]{babel} % English language/hyphenation
\usepackage{amsmath,amsfonts,amsthm} % Math packages

\usepackage{babelbib}

\usepackage{graphicx}
\usepackage{sectsty} % Allows customizing section commands
%\allsectionsfont{\centering \normalfont\scshape} % Make all sections centered, the default font and small caps
\usepackage{hyperref}

\usepackage{fancyhdr} % Custom headers and footers
\pagestyle{fancyplain} % Makes all pages in the document conform to the custom headers and footers
\fancyhead{} % No page header - if you want one, create it in the same way as the footers below
\fancyfoot[L]{} % Empty left footer
\fancyfoot[C]{} % Empty center footer
\fancyfoot[R]{\thepage} % Page numbering for right footer
\renewcommand{\headrulewidth}{0pt} % Remove header underlines
\renewcommand{\footrulewidth}{0pt} % Remove footer underlines
\setlength{\headheight}{13.6pt} % Customize the height of the header

\numberwithin{equation}{section} % Number equations within sections (i.e. 1.1, 1.2, 2.1, 2.2 instead of 1, 2, 3, 4)
\numberwithin{figure}{section} % Number figures within sections (i.e. 1.1, 1.2, 2.1, 2.2 instead of 1, 2, 3, 4)
\numberwithin{table}{section} % Number tables within sections (i.e. 1.1, 1.2, 2.1, 2.2 instead of 1, 2, 3, 4)

\setlength\parindent{0pt} % Removes all indentation from paragraphs - comment this line for an assignment with lots of text

%----------------------------------------------------------------------------------------
%	TITLE SECTION
%----------------------------------------------------------------------------------------

\newcommand{\horrule}[1]{\rule{\linewidth}{#1}} % Create horizontal rule command with 1 argument of height

\title{	
\normalfont \normalsize 
\textsc{Grafy i sieci} \\ [25pt] % Your university, school and/or department name(s)
\horrule{0.5pt} \\[0.4cm] % Thin top horizontal rule
\huge SK2. Wizualizacja algorytmów grafowych II \\ % The assignment title
\horrule{2pt} \\[0.5cm] % Thick bottom horizontal rule
\LARGE Dokumentacja wstępna
}%


\author{Michał Kielak, Michał Uziak} % Your name

\date{\normalsize\today} % Today's date or a custom date

\begin{document}

\maketitle % Print the title

%----------------------------------------------------------------------------------------
%	PROBLEM 1
%----------------------------------------------------------------------------------------

\section{Cel projektu}

Celem projektu jest implementacja oprogramowania do wizualizacji wybranych algorytmów kolorowania grafu.

\section{Algorytm}

W projekcie zostanie użyty algorytm Kruskala, wyznaczający minimalne drzewo rozpinające dla spójnego, ważonego grafu nieskierowanego. Algorytm został stworzony przez Josepha Kruskala w 1956 roku.
 

\subsection{Opis działania algorytmu Kruskala}

Kroki tworzenia drzewa w algorytmie Kruskala:
\begin{itemize}
	\item utworzenie lasu z wierzchołków (każdy wierzchołek jest osobnym drzewem) - zbiór L
	\item posortowanie krawędzi wg wag - zbiór S
	\item dopóki S>0
		\begin{itemize}
			\item wybranie i usunięcie krawędzi o najmniejszej wadze ze zbioru S
			\item jeśli krawędź łączyła dwa różne drzewa, dodanie jej do lasu L
			\item jeśli krawędź nie łączyła dwóch różnych drzew, usunięcie
		\end{itemize}
\end{itemize} 

\subsection{Złożoność obliczeniowa}

Przy obliczaniu złożoności obliczeniowej należy uwzględnić dwa etapy:
\begin{itemize}
	\item sortowanie krawędzi według wag - złożoność O(ElogV)
	\item O(E $\alpha$(E,V))
\end{itemize}
gdzie E - liczba punktów \\
V - liczba krawędzi \\
$\alpha$ - odwrotność funkcji Ackermanna

\section{Implementacja}

Projekt zostanie wykonany w języku Python, przy użyciu biblioteki Qt4. Założeniem programu jest działanie w czasie rzczywistym - użytkownik wprowadza przykładowy graf, następnie przechodząc przez kolejne ekrany może prześledzić kroki tworzenia minimalnego drzewa rozpinającego. Algorytm kończy działania, kiedy osiągnie minimalne drzewo rozpinające.
Przejście między kolejnymi krokami budowania drzewa będzie dla użytkownika niezauważalny, dlatego  czas wykonania algorytmu może zostać pominięty.

\subsection{Struktury danych}

Użytkownik będzie miał możliwość wczytania danych na dwa sposoby:

\begin{itemize}
\item przez graficzny interfejs użytkownika (rysowanie punktów, krawędzi, podawanie wag)
\item przez pliki tekstowe
\end{itemize}

\subsection{Format wejścia i wyjścia}

Format pliku wejściowego:

\begin{center}
MACIERZ WAG \\

1 1 1 0 \\
3 4 2 1
\end{center}

Format pliku wyjściowego:

\begin{center}
WSPÓŁRZĘDNE WIERZCHOŁKÓW \\
x y z \\
x y z
\end{center}

Po wykonaniu programu, użytkownik będzie miał możliwość zapisania otrzymanego grafu w postaci pliku graficznego.

\subsection{Sytuacje wyjątkowe}   

Możliwe sytuacje wyjątkowe:
\begin{itemize}
\item wczytanie źle sformatowanego pliku - pojawienie się okna z informacją o błędzie
\item połączenie dwóch tych samych wierzchołków więcej niż jedną krawędzia - zignorowane, wybranie pierwszej ze zdefiniowanych krawędzie
\item osierocone wierzchołki - usunięcie zbędnych wierzchołków
\end{itemize}

\subsection{Dodatkowe opcje}   

\begin{itemize}
\item kolorowanie krawędzi
\item ustawianie wielkości punktów
\item zapisywanie, jako pliki graficzne, kolejnych kroków tworzenia drzewa
\end{itemize}

\end{document}